% Template:     Artículo LaTeX
% Documento:    Archivo de ejemplo
% Versión:      1.2.5 (01/06/2023)
% Codificación: UTF-8
%
% Autor: Pablo Pizarro R.
%        pablo@ppizarror.com
%
% Manual template: [https://latex.ppizarror.com/articulo]
% Licencia MIT:    [https://opensource.org/licenses/MIT]

% ------------------------------------------------------------------------------
% NUEVA SECCIÓN
% ------------------------------------------------------------------------------
% Las secciones se inician con \section, si se quiere una sección sin número se
% pueden usar las funciones \sectionanum (sección sin número) o la función
% \sectionanumnoi para crear el mismo título sin numerar y sin aparecer en el índice
\section{Informes con \LaTeX}

% SUB-SECCIÓN
% Las sub-secciones se inician con \subsection, si se quiere una sub-sección
% sin número se pueden usar las funciones \subsectionanum (nuevo subtítulo sin
% numeración) o la función \subsectionanumnoi para crear el mismo subtítulo sin
% numerar y sin aparecer en el índice
\subsection{Una breve introducción}
	
	Este es un párrafo, puede contener múltiples \quotes{Expresiones} así como fórmulas o referencias\footnote{Las referencias se hacen utilizando la expresión \texttt{\textbackslash label}\{etiqueta\}.} como \eqref{eqn:identidad-imposible}. A continuación se muestra un ejemplo de inserción de imágenes (como la Figura \ref{img:testimage}) con el comando \href{https://latex.ppizarror.com/informe.html#hlp-imagen}{\textbackslash insertimage}:

	% Para insertar una imagen se puede usar la función \insertimage la cual
	% toma un primer parámetro opcional para definir una etiqueta (dentro de
	% los corchetes), luego toma la dirección de la imagen, sus parámetros
	% (en este caso se definió la escala de 0.16) y una leyenda opcional
	\insertimage[\label{img:testimage}]{test-image.png}{scale=0.16}{Where are you? de \quotes{Internet}.}

	A continuación\footnote{Como se puede observar las funciones \texttt{\textbackslash insert...} añaden un párrafo automáticamente.} se muestra un ejemplo de inserción de ecuaciones simples con el comando \href{https://latex.ppizarror.com/informe.html#hlp-formulae}{\textbackslash insertequation}:

	% Se inserta una ecuación, el primer parámetro entre [] es opcional
	% (permite identificar con una etiqueta para poder referenciarlo después
	% con \ref), seguido de aquello se escribe la ecuación en modo bruto sin signos $
	\insertequation[\label{eqn:identidad-imposible}]{\pow{a}{k}=\pow{b}{k}+\pow{c}{k} \quad \forall k>2}

	% Notar que no se requiere añadir un salto de línea después de insertar una imagen
	Este template ha sido diseñado para que sea completamente compatible con editores \LaTeX\ para escritorio y de manera online\scite{overleaf}. La compilación es realizada siempre usando las últimas versiones de las librerías, además se incluyen los parches oficiales para corregir eventuales \textit{warnings}. \\

	Este es un nuevo párrafo. Para crear un nuevo párrafo basta con usar \textbackslash\textbackslash\ en el anterior, lo que fuerza una nueva línea. También se insertar un nuevo párrafo con el comando \texttt{\textbackslash newp} si el compilador de latex arroja una alerta del tipo \textit{Underfull \textbackslash hbox (badness 10000) in paragraph at lines ...}

% SUB-SECCIÓN
\subsection{Añadiendo tablas}

	También puedes usar tablas, ¡Crearlas es muy fácil!. Puedes usar el plugin \href{https://www.ctan.org/tex-archive/support/excel2latex}{Excel2Latex} \cite{excel2latex} de Excel para convertir las tablas a \LaTeX\xspace o bien utilizar el \quotes{creador de tablas online} \cite{tablesgenerator}.

	% Tabla generada con el plugin Excel2Latex
	\begin{table}[H]
		\centering
		\caption{Ejemplo de tablas.}
		\begin{tabular}{ccc}
			\hline
			\textbf{Columna 1} & \textbf{Columna 2} & \textbf{Columna 3} \bigstrut\\
			\hline
			$\omega$ & $\nu$ & $\delta$ \bigstrut[t]\\
			$\Phi$ & $\Theta$ & $\varSigma$ \\
			$\xi$ & $\kappa$ & $\varpi$ \bigstrut[b] \\
			\hline
		\end{tabular}
		\label{tab:tabla-1}
	\end{table}


% ------------------------------------------------------------------------------
% NUEVA SECCIÓN
% ------------------------------------------------------------------------------
\section{Aquí un nuevo tema}

% SUB-SECCIÓN
\subsection{Haciendo reportes como un profesional}

	% Se inserta una imagen flotante en la izquierda del documento con
	% \insertimageleft, al igual que las demás funciones, el primer parámetro
	% es opcional, luego viene la ubicación de la imagen, seguido de la escala
	% (un 40% del ancho de página) y por último su leyenda. Para insertar una
	% imagen flotante en la derecha se utiliza \insertimageright usando los
	% mismos parámetros
	\insertimageleft[\label{img:imagen-izquierda}]{test-image-wrap}{0.4}{Apolo flotando a la izquierda.}

	~ \lipsum[1] \\

	% Párrafos de ejemplo
	\lipsum[2] \\
	\lipsum[3]

	% Agrega una ecuación con leyenda
	% Recordar que como no existe un índice, \insertindexequation no está disponible
	\insertequationcaptioned[\label{eqn:formulasinsentido}]{\int_{a}^{b} f(x) \dd{x} = \fracnpartial{f(x)}{x}{\eta} \cdotp \textstyle \sum_{x=a}^{b} f(x)\cancelto{1+\frac{\epsilon}{k}}{\bigp{1+\Delta x}}}{Ecuación sin sentido \cite{einstein}.}

	\lipsum[1]

% Inserta un subtítulo sin número
\subsection{Otros párrafos más normales}

	\begin{table*}
		\begin{threeparttable}
		\centering
		\caption{Ejemplo de tabla que usa múltiples columnas y tiene notas.}
		\begin{tabular}{cccc}
			\hline
			\textbf{Columna 1} & \textbf{Columna 2} & \textbf{Columna 3} & \textbf{Columna 4} \bigstrut\\
			\hline
			1 & $\omega$ & $\nu$ & $\delta$ \tnote{a} \\
			2 & $\Phi$ & $\Theta$ & $\varSigma$ \\
			3 & $\xi$ & $\kappa$ & $\varpi$ \\
			\hline
		\end{tabular}
		\begin{tablenotes}
			Esta tabla acepta comentarios y notas al margen.
			\item[a] Este elemento tiene una descripción debajo de la tabla
		\end{tablenotes}
		\end{threeparttable}
	\end{table*}

	% Párrafos
	\lipsum[7]

	% Se inserta una ecuación larga con el entorno gathered (1 solo número de ecuación)
	\insertgathered[\label{eqn:eqn-larga}]{
		\lpow{\Lambda}{f} = \frac{L\cdot f}{W} \cdot \frac{\pow{\lpow{Q}{e}}{2}}{8 \pow{\pi}{2} \pow{W}{4} g} + \sum_{i=1}^{l} \frac{f \cdot \bigp{M - d}}{l \cdot W} \cdot \frac{\pow{\bigp{\lpow{Q}{e}- i\cdot Q}}{2}}{8 \pow{\pi}{2} \pow{W}{4} g}\\
		Q_e = 2.5Q \cdot \int_{0}^{e} V(x) \dd{x} + \aasin{\biggp{1+\frac{1}{1-e}}}
	}

	% Nuevo párrafo
	\lipsum[4-5]
	
	% Ecuación encerrada en una caja
	\insertequation[]{ \boxed{f(x) = \fracdpartial{u}{t}} }
	
	% Párrafo 2
	\lipsum[6]
	
	% Inserta una imagen simple dentro del párrafo
	\insertimage{test-image-wrap}{width=4.5cm}{Imagen dentro de párrafo.}
	
	% Párrafo 3 del multicols
	\lipsum[11-13]

% SUB-SECCIÓN
\subsection{Ejemplos de inserción de código fuente}

	% A continuación se crea una función auxiliar, esta es una herramienta
	% extremadamente importante y muy útil. Esta función de ejemplo toma dos
	% parámetros, uno es el lenguaje del código fuente, el segundo el
	% identificador en el manual
	\newcommand{\insertsrcmanual}[2]{\href{https://latex.ppizarror.com/informe.html?srctype=#1\#hlp-srccode}{#2}}

	El template permite la inserción de los siguientes lenguajes de programación de forma nativa: \insertsrcmanual{abap}{ABAP}, \insertsrcmanual{ada}{Ada}, \insertsrcmanual{assemblerx64}{Assembler x64}, \insertsrcmanual{assemblerx86}{Assembler x86[masm]}, \insertsrcmanual{awk}{Awk}, \insertsrcmanual{bash}{Bash}, \insertsrcmanual{basic}{Basic}, \insertsrcmanual{c}{C}, \insertsrcmanual{caml}{Caml}, \insertsrcmanual{cmake}{CMake}, \insertsrcmanual{cobol}{Cobol}, \insertsrcmanual{cpp}{C++}, \insertsrcmanual{csharp}{C\#}, \insertsrcmanual{css}{CSS}, \insertsrcmanual{csv}{CSV}, \insertsrcmanual{cuda}{CUDA}, \insertsrcmanual{dart}{Dart}, \insertsrcmanual{docker}{Docker}, \insertsrcmanual{elisp}{Elisp}, \insertsrcmanual{elixir}{Elixir}, \insertsrcmanual{erlang}{Erlang}, \insertsrcmanual{fortran}{Fortran}, \insertsrcmanual{fsharp}{F\#}, \insertsrcmanual{glsl}{GLSL}, \insertsrcmanual{gnuplot}{Gnuplot}, \insertsrcmanual{go}{Go}, \insertsrcmanual{haskell}{Haskell}, \insertsrcmanual{html}{HTML}, \insertsrcmanual{ini}{INI}, \insertsrcmanual{java}{Java}, \insertsrcmanual{javascript}{Javascript}, \insertsrcmanual{json}{JSON}, \insertsrcmanual{julia}{Julia}, \insertsrcmanual{kotlin}{Kotlin}, \insertsrcmanual{latex}{LaTeX}, \insertsrcmanual{lisp}{Lisp}, \insertsrcmanual{llvm}{LLVM}, \insertsrcmanual{lua}{Lua}, \insertsrcmanual{make}{Make}, \insertsrcmanual{maple}{Maple}, \insertsrcmanual{mathematica}{Mathematica}, \insertsrcmanual{matlab}{Matlab}, \insertsrcmanual{mercury}{Mercury}, \insertsrcmanual{modula2}{Modula-2}, \insertsrcmanual{objectivec}{Objective-C}, \insertsrcmanual{octave}{Octave}, \insertsrcmanual{opencl}{OpenCL}, \insertsrcmanual{opensees}{OpenSees}, \insertsrcmanual{pascal}{Pascal}, \insertsrcmanual{perl}{Perl}, \insertsrcmanual{php}{PHP}, \insertsrcmanual{plaintext}{Texto plano}, \insertsrcmanual{postscript}{PostScript}, \insertsrcmanual{powershell}{Powershell}, \insertsrcmanual{prolog}{Prolog}, \insertsrcmanual{promela}{Promela}, \insertsrcmanual{pseudocode}{Pseudocódigo}, \insertsrcmanual{python}{Python}, \insertsrcmanual{qsharp}{Q\#}, \insertsrcmanual{r}{R}, \insertsrcmanual{racket}{Racket}, \insertsrcmanual{reil}{Reil}, \insertsrcmanual{ruby}{Ruby}, \insertsrcmanual{rust}{Rust}, \insertsrcmanual{scala}{Scala}, \insertsrcmanual{scheme}{Scheme}, \insertsrcmanual{scilab}{Scilab}, \insertsrcmanual{simula}{Simula}, \insertsrcmanual{sparql}{SPARQL}, \insertsrcmanual{sql}{SQL}, \insertsrcmanual{swift}{Swift}, \insertsrcmanual{tcl}{TCL}, \insertsrcmanual{vbscript}{VBScript}, \insertsrcmanual{verilog}{Verilog}, \insertsrcmanual{vhdl}{VHDL} y \insertsrcmanual{xml}{XML}. \\

	Para insertar un código fuente se debe usar el entorno \texttt{sourcecode}, o el entorno \texttt{sourcecodep} si es que se quiere utilizar parámetros adicionales. A continuación se presenta un ejemplo de inserción de código fuente en Python (Código \ref{codigo-python}), Java y Matlab:

% Se define el lenguaje del código. Cuidado: Los códigos en LaTeX son sensibles
% a las tabulaciones y espacios en blanco
\begin{sourcecode}[\label{codigo-python}]{python}{Ejemplo en Python.}
import numpy as np

def incmatrix(genl1, genl2):
	m = len(genl1)
	n = len(genl2)
	M = None # Comentario 1
	VT = np.zeros((n*m, 1), int) # Comentario 2
\end{sourcecode}

\begin{sourcecode}[]{java}{Ejemplo en Java.}
import java.io.IOException;
import javax.servlet.*;

// Hola mundo
public class Hola extends GenericServlet {
	public void service(ServletRequest request, ServletResponse response)
	throws ServletException, IOException{
		response.setContentType("text/html");
		PrintWriter pw = response.getWriter();
		pw.println("Hola, mundo!");
	}
}
\end{sourcecode}

\begin{sourcecode}{matlab}{Ejemplo en Matlab.}
% Se crea gráfico
f = figure(1);

hold on;
movegui(f, 'center');

xlabel('td/Tn');
ylabel('FAD=Umax/Uf0');

for j = 1:length(BETA)
	fad = ones(1, NDATOS); % Arreglo para el FAD
	for i = 1:NDATOS
		[t, u_t, ~, ~] = main(BETA(j), r(i), M, K, F0, 0);
		t2 = t * cos(r(i)) ./ BETA;
		fad(i) = max(abs(u_t)) / uf0;
		BETA(j) *= 0.6;
		fad(i) = fad(i) .* t2;
	end
end
\end{sourcecode}

\begin{sourcecode}{pseudocodecolor}{Algoritmo Non-Max supression.}
function NMS(B,c)
	$B_{nms} \leftarrow \emptyset$
	
	for $b_i \in B$ do
		$discard \leftarrow$ False
		for $b_j \in B$ do
			if same($b_i$,$b_j$) > $\lambda_{nms}$ then
				if score($c$,$b_j$) > score($c$, $b_i$) then
					$discard \leftarrow$ True
		if not $discard$ then
			$B_{nms} \leftarrow B_{nms} \bigcup b_i$
	
	return $B_{nms}$
\end{sourcecode}

% SUB-SECCIÓN
\subsection{Agregando múltiples imágenes}

	El template ofrece el entorno \href{https://latex.ppizarror.com/informe.html#hlp-images}{images} que permite insertar múltiples imágenes de una manera muy sencilla. Para crear imágenes múltiples se deben usar las siguientes instrucciones:

\begin{sourcecode}{latex}{}
\begin{imagesmc}{top}{Ejemplo de imagen múltiple que usa todo el ancho de la página y esta en la parte superior de la misma.}
	\addimage[\label{ciudadfoto}]{test-image}{height=4.5cm}{Ciudad}
	\addimageanum{test-image-wrap}{height=4.5cm}
\end{imagesmc}
\end{sourcecode}

	Obteniendo así:

	\begin{imagesmc}{top}{Ejemplo de imagen múltiple que usa todo el ancho de la página y esta en la parte superior de la misma.}
		\addimage{test-image}{height=4.5cm}{Ciudad}
		\addimageanum{test-image-wrap}{height=4.5cm}
	\end{imagesmc}


% ------------------------------------------------------------------------------
% NUEVA SECCIÓN
% ------------------------------------------------------------------------------
% Inserta una sección sin número
\sectionanum{Más ejemplos}

% Inserta un subtítulo sin número
\subsectionanum{Listas y Enumeraciones}

	Hacer listas enumeradas con \LaTeX\ es muy fácil con el template\footnote{También puedes revisar el manual de las enumeraciones en \url{https://latex.ppizarror.com/doc/enumitem.pdf}.}, para ello debes usar el comando \texttt{\textbackslash begin\{enumerate\}}, cada elemento comienza por \texttt{\textbackslash item}, resultando así:

	\begin{enumerate}
		\item Grecia
		\item Abracadabra
		\item Manzanas
	\end{enumerate}

	También se puede cambiar el tipo de enumeración, se pueden usar letras, números romanos, entre otros. Esto se logra cambiando el \textbf{label} del objeto \texttt{enumerate}. A continuación se muestra un ejemplo usando letras con el estilo \texttt{\textbackslash alph}\footnote{Con \texttt{\textbackslash Alph} las letras aparecen en mayúscula.}, números romanos con \texttt{\textbackslash roman}\footnote{Con \texttt{\textbackslash Roman} los números romanos salen en mayúscula.} o números griegos con \texttt{\textbackslash greek}\footnote{Una característica propia del template, con \texttt{\textbackslash Greek} las letras griegas están escritas en mayúscula.}:

	\begin{multicols}{3}
		\begin{enumeratebf}[label=\alph*) ] % Fuente en negrita
			\item Peras
			\item Manzanas
			\item Naranjas
		\end{enumeratebf}

		\begin{enumerate}[label=\roman*) ]
			\item Rojo
			\item Café
			\item Morado
		\end{enumerate}
		
		\begin{enumerate}[label=\greek*) ]
			\item Historia
			\item Lenguaje
			\item Filosofía
		\end{enumerate}
	\end{multicols}

	Para hacer listas sin numerar con \LaTeX\ hay que usar el comando \texttt{\textbackslash begin\{itemize\}}, cada elemento empieza por \texttt{\textbackslash item}, resultando:

	\begin{multicols}{3}
		\begin{itemize}[label={--}]
			\item Peras
			\item Manzanas
			\item Naranjas
		\end{itemize}

		\begin{enumerate}[label={*}]
			\item Rojo
			\item Café
			\item Morado
		\end{enumerate}

		\begin{itemize}
			\item Árboles
			\item Pasto
			\item Flores
		\end{itemize}
	\end{multicols}

% Inserta un subtítulo sin número
\subsectionanum{Otros}

	Recuerda revisar el manual de todas las funciones y configuraciones de este template visitando el siguiente link: \url{https://latex.ppizarror.com/reporte} \cite{template}. Si encuentras un error en el template, puedes abrir un nuevo issue a través de su página en GitHub.


% ------------------------------------------------------------------------------
% REFERENCIAS, revisar configuración \stylecitereferences
% ------------------------------------------------------------------------------
\bibliography{library}


% ------------------------------------------------------------------------------
% ANEXO
% ------------------------------------------------------------------------------
\begin{appendixd}

	\section{Cálculos realizados}

	\subsection{Metodología}
	\lipsum[1-2]

	% Imagen, se numerará automáticamente con la letra del anexo según
	% la configuración \appendixindepobjnum
	\insertimage[\label{img:anexo-2}]{test-image.png}{width=1\linewidth}{Imagen en anexo.}

	\subsection{Resultados}
	\lipsum[10]

	% Tablas
	\enabletablerowcolor[2] % Activa el color de celda
	\begin{table}[H]
		\centering
		\caption{Tabla de cálculo.}
		\begin{tabular}{ccc}
			\hline
			\textbf{Elemento} & $\epsilon_i$ & \textbf{Valor} \bigstrut\\
			\hline
			A     & 10    & 3,14$\pi$ \\
			B     & 20    & 6 \\
			C     & 30    & 7 \\
			D     & 150    & 10 \\
			E     & 0    & 0 \\
			\hline
			\end{tabular}
		\label{tab:anexo-1}
	\end{table}
	\disabletablerowcolor % Desactiva el color de celda

	\section{Más cálculos}

	% Párrafo
	\lipsum[1] \\
	\lipsum[4]

	% Tabla de encuestas
	\begin{table}[H]
		\centering
		\caption{Resultados encuesta.}
		\begin{tabular}{ccc}
			\hline
			\textbf{Herramienta} & \textbf{Nota} & \textbf{Recomendado} \bigstrut\\
			\hline
			\LaTeX & 100\% & Si $\checkmark$ \\
			Microsoft Word \textsuperscript{\textregistered} & 0\% & No $\frownie$\\
			\hline
		\end{tabular}
		\label{tab:anexo-2}
	\end{table}

\end{appendixd}